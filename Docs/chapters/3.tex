\section{Datos}
Los datos proporcionados contienen 55 variables, con 103230 observaciones para el entrenamiento del modelo y 5618 observaciones para la entrega de predicciones.
\par
\subsection{Origen de los datos}
Estos datos se han obtenido al juntar los datos satelitales con los del catastro, suponemos que se han unido a través de las cordenadas X, Y.
\subsection{Variables}
\begin{itemize}
\item Cordenadas
\par
Son X y Y, en los datos catastrales corresponden con los centroides de las parcelas\cite{catastro}
\item Imagenes Satelitales
\par
Valores tomados de 4 bandas: rojo(4), verde(3), azul(2) y infrarroja(8). Todas con resolución espacial de 10m.
\par
Para cada banda se definen 10 valores
\item Area
\par
Superficie en $m^{2}$ de la parcela
\item Geom
\par
Vertices?
\item Construction year
\par
Año de construcción de la parcela
\item Max. Bulding Floor
\par
Número del último piso de la parcela
\item Cadastral Quality ID
\par
Calidad de la construcción de la parcela. 
Viene indicada por el último dígito de la tipología. 
Toma valores de 1 a 9, yendo de mejor a peor. Excepcionalmente, puede aparecer como A, B y C para lo mejor que lo mejor\cite{catastro}



\end{itemize}
\subsection{Missing values}
Hay 40 valores nulos en el train set y 14 valores nulos en el test set.
En ambos casos la cantidad de observaciones con valores nulos es exactamente la mitad, 20 y 7 respectivamente.
Esto es porque solo hay dos variables que contengan valores nulos y no existe ninguna observación en la que solo de dichas variables sea nula.
\begin{itemize}
    \item Max. Bulding Floor
    \item Cadastral Quality ID
\end{itemize}




\subsection{Distribución test y train}